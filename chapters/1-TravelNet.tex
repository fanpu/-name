\begin{savequote}[75mm] 
You must be shapeless, formless, like water. When you pour water in a cup, it becomes the cup. When you pour water in a bottle, it becomes the bottle. When you pour water in a teapot, it becomes the teapot. Water can drip and it can crash. Become like water my friend.
\qauthor{Bruce Lee} 
\end{savequote}

\chapter{TravelNet}

\newthought{\name is an on-line platform } that connects travellers to local guides, with the vision to provide each traveller a unique and personalized experience. Local guides draw out travel plans, which include accommodations, transport and tourist destinations, and travellers choose the plan that best suit them. It is innovative due to the high level of personalization and intimacy that locals are able to offer to their guests as compared to tours by travel agencies. The relatively low price of such tours is a key selling point as well.


Currently tours are provided to travellers by tour agencies, which then assign tour guides to bring them around various locales according to a set itinerary. 

Our goal is to link locals with tourists, to provide a customized tour for the tourists just how they like it, with a guide that is an everyday citizen from the area. This way the tours will be more informal and personalized, closer to friends enjoying the view together rather than a tour guide narrating to a tourist. Tours will also be more personal, as it will be a one-to-one experience rather than the one-to-many tours normally available.

Not only that, tourists will be able to pick their own guides. Guides and tourists will be reviewed or "Starred"\footnote {See Sect. 5 for more details}. Based on past experiences by others, both tourists and guides will be able to choose their partners, resulting in more pleasant journeys together.


\section{Business Model}

Some stuff taken from wiki to take note of; delete after writing:

A business model describes the rationale of how an organization creates, delivers, and captures value, in economic, social, cultural or other contexts. The process of business model construction is part of business strategy.

In an earlier article on Harvard Business Review, Sangeet Paul Choudary elaborates on the three elements of a successful platform business model. \textbf{The Toolbox} creates connection by making it easy for others to plug into the platform. This infrastructure enables interactions between participants. \textbf{The Magnet} creates pull that attracts participants to the platform. For transaction platforms, both producers and consumers must be present to achieve critical mass. \textbf{The Matchmaker} fosters the flow of value by making connections between producers and consumers. \emph{Data} is at the heart of successful matchmaking, and distinguishes platforms from other business models.

Chen (2009) pointed out that the business model in the twenty-first century has to \uwave{take into account the capabilities of Web 2.0, such as collective intelligence, network effects, user generated content, and the possibility of self-improving systems.} He suggested that the service industry such as the airline, traffic, transportation, hotel, restaurant, Information and Communications Technology and Online gaming industries will be able to benefit in adopting business models that take into account the characteristics of Web 2.0. He also emphasized that Business Model 2.0 has to take into account not just the technology effect of Web 2.0 but also the networking effect. He gave the example of the success story of Amazon in making huge profits each year by developing a full blown open platform that supports a large and thriving community of companies that re-use Amazon’s On Demand commerce services.


\section{Minimum Viable Product}
